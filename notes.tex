\documentclass[12pt,a4paper]{book}
\usepackage[utf8]{inputenc}
\usepackage{amsmath}
\usepackage{amssymb}
\usepackage{hyperref}

\newcommand{\Px}{\mathbb{P}}
\newcommand{\E}{\mathbb{E}}
\newcommand{\logit}{\text{logit}}
\newcommand{\R}{\mathbb{R}}

\title{Dentsu Aegis Data Event}
\author{Ryerson Applied Math M.Sc Group}
\date{January 30, 2020}

\begin{document}

\maketitle
\tableofcontents

\chapter{Preliminary Thoughts}

\section{Introduction}
I set this up for us to have a place to write any notes before going into the event and during so we can share any helpful tool and resources we have. From the little reading I've done it seems like we'll be looking at an online marketing exercise. Most of the services they've provide have been using AI and ML algorithms in order to provide targeted ads (eg. using face recognition on social media to determine which people have pets and push pet food ads through their feed). \\
\begin{flushright}
- Alin
\end{flushright}

\section{Logistic Regression}

A very well known quantity in online marketing campaigns is logistic regression. It uses the logistic function in order to determine the likelihood of success (ad being appropriate for a given person) and then based on a threshold makes a binary decision on whether or not you should push the ad out to the user. This can be extended to a multinomial logistic regression where we can compute likelihoods for multiple categories (use threshold to say yes, maybe, no) on the ad and then perform a constraint optimization factoring in costs of pushing ads. 

The model is set up as follows. 
\begin{itemize}
    \item Let $p = \Px (X=1)$ be the probability of a success 
    \item Let $\{ x_1, \ldots, x_n \}$ be the covariates 
    \item Let $\beta_0, \beta_1, \ldots, \beta_n$ be the coefficients
    \item note that $\logit: \R \rightarrow (0,1)$ is defined as $\logit(x) = \frac{1}{1-e^{-x}}$
\end{itemize}

With the regression equation being

\begin{equation}
    \logit(p) = \beta_o + \beta_1 x_1 + \ldots \beta_n x_n
\end{equation}

Fitting of this model can be done in several ways with the most common being the maximum likelihood estimator (MLE). An example of \href{https://stats.idre.ucla.edu/r/dae/logit-regression/}{logistic regression in R} via the \texttt{glm} function along with several tests for assessing the quality of the fit. 

\section{Possible "Solutions"}

Below are two standard applications of ML in digital marketing. Several algorithms can be used to solve these problems which will be detailed as needed but in the simplest case can be solved with the logistic regression model above. 

\subsection{Content Curation}
This problem is one of filtering. Provided with some information about a user, we should be able to eliminate "annoying" content/ads from their experience while at the same time providing ads that are most likely to drive business. 

\subsection{Product Recommendation}
In simple terms, this is a matching exercise. Provided with some information about a client, we should be able to offer some products that they are more likely to be interested in. Solving this problem can be done using a multinomial logistic regression where the dimension of $p$ (the vector containing probabilities of "success") is equal to the number of items in your inventory. The model should be trained on previous purchasing data. It is worth noting that majority of the covariates will be factor variables so regression may appear to be a poor fit. 

\chapter{Data Work}


\section{Background}
We have a cannabis brand looking at the effects of legalization on cannabis consumption and the consumer attitude within that. The goal is to create a narrative around this idea that can be given to stakeholders. \\
\textbf{Contact:} \href{mailto: }{email} by 1:00pm to confirm participation 

\subsection{Initial Reaction}
At first glimpse, this project seems like it should be approached as a \textit{sentiment analysis}. We should try to do some text mining in order to augment the data given already. Here is an example of a \href{https://www.tidytextmining.com/sentiment.html}{sentiment analysis} in R using tidy data (this is a clear definition which will guide the data cleaning process).


\section{The Dataset}
THe data comes in several files looking at different aspects of the cannabis market. These will be combined in order to give insight into how the market feels about the product and potential habbits. 

\section{EDA}

\section{Insight and Solutions}
To be done in R Markdown or Jupyter Notebooks. Here is a an extra step.
\end{document}
